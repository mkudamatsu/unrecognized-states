\documentclass[12pt,a4paper]{article}%
\usepackage{amsmath}
\usepackage{amsfonts}
\usepackage{amssymb}
\usepackage{graphicx}%
\setcounter{MaxMatrixCols}{30}
%TCIDATA{OutputFilter=latex2.dll}
%TCIDATA{Version=5.50.0.2953}
%TCIDATA{CSTFile=40 LaTeX article.cst}
%TCIDATA{Created=Saturday, March 08, 2008 16:13:47}
%TCIDATA{LastRevised=Monday, March 10, 2008 16:21:31}
%TCIDATA{<META NAME="GraphicsSave" CONTENT="32">}
%TCIDATA{<META NAME="SaveForMode" CONTENT="1">}
%TCIDATA{BibliographyScheme=Manual}
%TCIDATA{<META NAME="DocumentShell" CONTENT="Standard LaTeX\Blank - Standard LaTeX Article">}
%BeginMSIPreambleData
\providecommand{\U}[1]{\protect\rule{.1in}{.1in}}
%EndMSIPreambleData
\newtheorem{theorem}{Theorem}
\newtheorem{acknowledgement}[theorem]{Acknowledgement}
\newtheorem{algorithm}[theorem]{Algorithm}
\newtheorem{axiom}[theorem]{Axiom}
\newtheorem{case}[theorem]{Case}
\newtheorem{claim}[theorem]{Claim}
\newtheorem{conclusion}[theorem]{Conclusion}
\newtheorem{condition}[theorem]{Condition}
\newtheorem{conjecture}[theorem]{Conjecture}
\newtheorem{corollary}[theorem]{Corollary}
\newtheorem{criterion}[theorem]{Criterion}
\newtheorem{definition}[theorem]{Definition}
\newtheorem{example}[theorem]{Example}
\newtheorem{exercise}[theorem]{Exercise}
\newtheorem{lemma}[theorem]{Lemma}
\newtheorem{notation}[theorem]{Notation}
\newtheorem{problem}[theorem]{Problem}
\newtheorem{proposition}[theorem]{Proposition}
\newtheorem{remark}[theorem]{Remark}
\newtheorem{solution}[theorem]{Solution}
\newtheorem{summary}[theorem]{Summary}
\newenvironment{proof}[1][Proof]{\noindent\textbf{#1.} }{\ \rule{0.5em}{0.5em}}

\usepackage{bigfoot} % Allow \verb in a footnote

%%%%%%%%%%%%%%%%%%%%%%%%%%%%%%%%%%%%%%%%%%%%%%%%%%%%%%%%%%%%%%%%%%%%%%%%%%%%%%%%%%%%%%%
\begin{document}

\title{Estimating Output Loss due to being Unrecognized as States}
\date{}
\maketitle

\section{Introduction}
\section{Data}
\subsection{Real GDP}
We use the Penn World Table 9.0 (Freenstra et al.\ 2015), as the data source for real GDP. 
Expenditure-side real GDP at chained PPPs in 2011 US dollars (the variable \textit{rgdpe}) is used because the variable is meant for comparing ``living standards across countries and across years'' (Freenstra et al.\ 2015, Table 1).

% Comparison to WDI
An alternative data source for real GDP is World Development Indicators (World Bank 2018). 
We prefer the Penn World Table, however, because since version 8 it makes real GDP cross-sectionally comparable in multiple years. 
The cross-sectional comparability in the World Development Indicators is only ensured for the latest international price survey year (2010 for the latest version); the farther away from this reference year, the less comparable its data on real GDP across countries (see Pinkovskiy and Sala-i-Martin 2016, sections 2.1-2.2, for detail).
As we exploit the variation in nighttime light not only within each country but also within each year, cross-sectional comparability is important.

\subsection{Nighttime light}
The data source for nighttime light is the DMSP-OLS Nighttime Lights Time Series, Version 4 (National Geophysical Data Center 2015). 
It provides annual panel data of nighttime light intensity at the 30 arc-second cells across the world. 
The value of light intensity ranges from 0 to 63, which is not comparable across years due to the aging and replacement of satellite sensors. 
See Henderson et al.\ (2012) among others for more detail about the dataset.
Following Hodler and Raschky (2014), who aggregate light data into the sub-national level, we take the average of all the cell-level values within each country/territory as a measure of light intensity at the country/territory level.

% Validation
Many studies validate light intensity against the measures of living standards.%
\footnote{See Mellander et al.\ (2015), Bickenbach et al.\ (2016), and Goldblatt et al.\ (2018) for recent validation exercises at the sub-national level.} 
For validation against real GDP at the country level, Henderson et al.\ (2012) and Storeygard (2016, table 1) show that the elasticity of real GDP with respect to light is around 0.3 across different subsamples of countries. 
In the estimation of elasticity, they control for country and year fixed effects, the latter of which take into account that the value of light intensity is not comparable across years.

\subsection{Country boundaries}
To aggregate nighttime light intensity at the 30 arc-second cell level into the country level, we need the spatial data on country boundaries. 
We use GADM (2018), following Alesina et al.\ (2016) and Dreher et al.\ (2015) among others. 
The data include 256 countries and overseas territories. 
One country, Vatican, is dropped from the sample because it is smaller than the 30 arc-second cell. 
In addition, 73 countries and territories are also dropped from the sample as real GDP data is unavailable.

\section{Methodology}
% Model of real GDP
We model real GDP in territory $i$ in year $t$, denoted by $y_{it}$, as follows:
\begin{equation}\label{gdp}
y_{it} = \beta L_{it} + \mu_i + \eta_t + \varepsilon_{it},
\end{equation}
where $L_{it}$ is nighttime light intensity in territory $i$ in year $t$, $\mu_i$ the territory fixed effect, $\eta_t$ the year fixed effect, and $\varepsilon_{it}$ the error term.

% Sample
To estimate $\beta$, $\mu_i$, and $\eta_t$ in equation (\ref{gdp}) with OLS, we use the balanced panel of countries around the world from 1992 to 2013 so that we can avoid compositional bias in the estimation of territory and year fixed effects, $\mu_i$ and $\eta_t$. 
The sample of countries excludes the parent countries of unrecognized states (i.e.\ Georgia, Azerbaijan, and Moldova) because we will validate our predicted values of real GDP against the actual real GDP in these three countries.
Standard errors are clustered both at the territory level and at the year level with the multi-way cluster robust inference method (Cameron et al.\ 2011).

% Model of country fixed effects
To predict $y_{it}$ for unrecognized states (and their parent countries dropped from the sample) from equation (\ref{gdp}), we need to recover territory fixed effects, $\mu_i$. 
We predict them with the average light intensity over the sample period of 1992-2013. 
Specifically, with the same set of countries as the one for estimating equation (\ref{gdp}), we estimate the following equation with OLS:
\begin{equation}\label{country_fe}
\hat{\mu}_i = \alpha + \gamma \bar{L}_{i} + \xi_{i},
\end{equation}
where $\hat{\mu}_i$ is the estimated territory fixed effect from equation (\ref{gdp}), $\bar{L}_{i}$ the average light intensity in territory $i$ over the sample period, and $\xi_i$ the error term.

% Prediction of real GDP
Consequently, GDP per capita in unrecognized state $k$ in year $t$, denoted by $\hat{y}_{kt}$, will then be predicted as follows:
\begin{equation}
\hat{y}_{kt} = \hat{\beta} L_{kt} + [\hat{\alpha} + \hat{\gamma} \bar{L}_{i}]  + \hat{\eta_t},
\end{equation}
where $\hat{\beta}$ and $\hat{\eta_t}$ are the estimated $\beta$ and $\eta_t$ in equation (\ref{gdp}), $\hat{\alpha}$ and $\hat{\gamma}$ the estimated $\alpha$ and $\gamma$ in equation (\ref{country_fe}).

\section{Results}
\section{Conclusions}


\begin{thebibliography}{99}                                                                                                %
\bibitem{} Alesina, Alberto, Stelios Michalopoulos, and Elias Papaioannou. 2016. ``Ethnic Inequality.'' \textit{Journal of Political Economy}, 124(2): 428-488.
\bibitem{} Bickenbach, Frank, Eckhardt Bode, Peter Nunnenkamp, and Mareike Soder. 2016. ``Night lights and regional GDP.'' \textit{Review of World Economics}, 152 (2), 425-447.                                                                                         
\bibitem{} Cameron, A. Colin, Jonah B. Gelbach, and Douglas L. Miller. 2011. ``Robust Inference With Multiway Clustering.'' \textit{Journal of Business \& Economic Statistics}, 29(2): 238-249.
\bibitem{} Dreher, Axel, Andreas Fuchs, Roland Hodler, Bradley C. Parks, Paul A. Raschky, and Michael J. Tierney. 2015. ``Aid on Demand: African Leaders and the Geography of China?s Foreign Assistance.'' CEPR Discussion Paper, 10170.
\bibitem{} Feenstra, Robert C., Robert Inklaar, and Marcel P. Timmer. 2015. ``The Next Generation of the Penn World Table.'' \textit{American Economic Review}, 105(10): 3150-3182.
\bibitem{} GADM. 2018. ``GADM Maps and Data (version 3.6).'' https://gadm.org (accessed February 7, 2019).
\bibitem{} Goldblatt, Ran, Kilian Heilmann, and Yonatan Vaizman. 2018. ``Can Medium-Resolution Satellite Imagery Measure Economic Activity at Small Geographies? Evidence from Landsat in Vietnam.'' \textit{World Bank Economic Review}, forthcoming.
\bibitem{} Henderson, J. Vernon, Adam Storeygard, and David N. Weil. 2012. ``Measuring Economic Growth from Outer Space.'' \textit{American Economic Review}, 102(2): 994-1028.
\bibitem{} Hodler, Roland, and Paul A. Raschky. 2014. ``Regional Favoritism.'' \textit{Quarterly Journal of Economics}, 129(2): 995-1033.
\bibitem{} Mellander, Charlotta, Jose Lobo, Kevin Stolarick, and Zara Matheson. 2015. ``Night-Time Light Data: A Good Proxy Measure for Economic Activity?'' \textit{PloS one}, 10(10), e0139779.
\bibitem{} National Geophysical Data Center. 2015. ``Version 4 DMSP-OLS Nighttime Lights Time Series.'' National Oceanic and Atmospheric Administration. https://ngdc.noaa.gov/eog/dmsp/downloadV4composites.html (accessed February 6, 2019).
\bibitem{} Pinkovskiy, Maxim, and Xavier Sala-i-Martin. 2016. ``Newer Need Not be Better: Evaluating the Penn World Tables and the World Development Indicators Using Nighttime Lights.''. NBER Working Paper, 22216.
\bibitem{} Storeygard, Adam. 2016. ``Farther on down the Road: Transport Costs, Trade and Urban Growth in Sub-Saharan Africa.'' \textit{The Review of Economic Studies}, 83(3): 1263-1295.
\bibitem{} World Bank. 2018. \textit{World Development Indicators}. http://wdi.worldbank.org (accessed February 27, 2019).
\end{thebibliography}

\end{document}