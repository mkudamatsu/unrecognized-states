\documentclass[12pt,a4paper]{article}%
\usepackage{amsmath}
\usepackage{amsfonts}
\usepackage{amssymb}
\usepackage{graphicx}%
\setcounter{MaxMatrixCols}{30}
%TCIDATA{OutputFilter=latex2.dll}
%TCIDATA{Version=5.50.0.2953}
%TCIDATA{CSTFile=40 LaTeX article.cst}
%TCIDATA{Created=Saturday, March 08, 2008 16:13:47}
%TCIDATA{LastRevised=Monday, March 10, 2008 16:21:31}
%TCIDATA{<META NAME="GraphicsSave" CONTENT="32">}
%TCIDATA{<META NAME="SaveForMode" CONTENT="1">}
%TCIDATA{BibliographyScheme=Manual}
%TCIDATA{<META NAME="DocumentShell" CONTENT="Standard LaTeX\Blank - Standard LaTeX Article">}
%BeginMSIPreambleData
\providecommand{\U}[1]{\protect\rule{.1in}{.1in}}
%EndMSIPreambleData
\newtheorem{theorem}{Theorem}
\newtheorem{acknowledgement}[theorem]{Acknowledgement}
\newtheorem{algorithm}[theorem]{Algorithm}
\newtheorem{axiom}[theorem]{Axiom}
\newtheorem{case}[theorem]{Case}
\newtheorem{claim}[theorem]{Claim}
\newtheorem{conclusion}[theorem]{Conclusion}
\newtheorem{condition}[theorem]{Condition}
\newtheorem{conjecture}[theorem]{Conjecture}
\newtheorem{corollary}[theorem]{Corollary}
\newtheorem{criterion}[theorem]{Criterion}
\newtheorem{definition}[theorem]{Definition}
\newtheorem{example}[theorem]{Example}
\newtheorem{exercise}[theorem]{Exercise}
\newtheorem{lemma}[theorem]{Lemma}
\newtheorem{notation}[theorem]{Notation}
\newtheorem{problem}[theorem]{Problem}
\newtheorem{proposition}[theorem]{Proposition}
\newtheorem{remark}[theorem]{Remark}
\newtheorem{solution}[theorem]{Solution}
\newtheorem{summary}[theorem]{Summary}
\newenvironment{proof}[1][Proof]{\noindent\textbf{#1.} }{\ \rule{0.5em}{0.5em}}

\usepackage{bigfoot} % Allow \verb in a footnote

% Make figure captions left-aligned: see https://tex.stackexchange.com/questions/275131/align-caption-to-the-left
\usepackage{caption} 
\captionsetup{justification=raggedright,singlelinecheck=false}

%%%%%%%%%%%%%%%%%%%%%%%%%%%%%%%%%%%%%%%%%%%%%%%%%%%%%%%%%%%%%%%%%%%%%%%%%%%%%%%%%%%%%%%
\begin{document}

\title{Recovering Economic Growth Rates \\ of Unrecognized States \\ in the Former Soviet Union \\ from Nighttime Light \thanks{I thank Elisa Kugaevskaya for her superb research assistance on this project. Financial support from Japan Society for the Promotion of Science (JSPS Kakenhi Grant Number 17K13730) is gratefully acknowledged.} }
\author{Masayuki Kudamatsu \thanks{Osaka School of International Public Policy, Osaka University. Email: kudamatsu@osipp.osaka-u.ac.jp}}
\date{\today}
\maketitle

\section{Introduction}
Back in 1991, the Soviet Union broke up into 15 constituent republics. Immediately after these republics became independent, wars of secession broke out in three of them: Azerbaijan, Georgia, and Moldova. Secessionists, consisting of minority ethnic groups in each republic, won the wars and declared the independence of their territory: Nagorno-Karabakh in Azerbaijan, Abkhazia and South Ossetia in Georgia, and Transnistria in Moldova. However, not a single country around the world recognized them as sovereign states. These four ``states'' remain internationally unrecognized until today.%
\footnote{%
	Since 2008, Russia and a few other countries have recognized Abkhazia and South Ossetia as a sovereign state, though.
} 

This paper attempts to estimate the impact of being unrecognized as a sovereign state on economic activities in these four territories of the former Soviet Union. The empirical challenge for such endeavor is the lack of data. International organizations such as the World Bank do not provide any data for unrecognized states. The limited state capacity of these territories implies that there is no official statistics. Any surveys conducted in their parent countries exclude these territories from the sampling procedure.%
\footnote{%
	As an example, the Demographic and Health Survey conducted in Moldova in 2005 excludes the entire Transnistria from its sample (National Scientific and Applied Center for Preventive Medicine 2006: xxvi).
} 
To overcome this data limitation, the present study uses the satellite images of nighttime light for the period 1993-2013 to recover the unobserved annual economic growth rates of the four unrecognized states in the former Soviet Union.

Witmer and O'Loughlin (2011) estimate the impact of wars in the caucasus regions by using nighttime light data.

\section{Background}
\subsection{Nagorno-Karabakh}
Nagorno-Karabakh is a mountainous enclave of Armenian people in the west of Azerbaijan. Armenians are culturally different from Azerbaijanis in terms of religion (Armenian Orthodox Christianity versus Islam, mostly Shia) and language (Armenian vs Turkish). 

In 1923, the Soviet Union's authority drew the border of Nagorno-Karabakh Autonomous Oblast from scratch, based on their understanding of where Armenians resided within Azerbaijan.%
\footnote{
	An autonomous oblast is a district with some degree of autonomy from the government of a constituent republic of the Soviet Union. 
} 
The Soviet leaders made Nagorno-Karabakh part of Azerbaijan, rather than of Armenia, for two reasons, according to de Waal (2013: 144-145). First, Moscow needed to appease Azeri politicians to control their oil fields in the Caspian Sea. Second, Armenians in Nagorno-Karabakh economically relied on the eastern part of Azerbaijan, and the communist ideal dictated that economic consideration should override ethnic integration.

During the Soviet days, the Azerbaijani authority encouraged the migration of Azeris into Nagorno-Karabakh. Population of Azerbaijanis in Nagorno-Karabakh jumped up from 13,000 in 1926 to 37,000 in 1979 while the corresponding figures for Armenians are 117,000 and 123,000, respectively (de Waal 2013: 153-154). The Azeri authority also ignored the Armenian culture in Nagorno-Karabakh: there was no Armenian-language television channel; the history of Armenia not taught in Armenian-language schools; medieval Armenian churches closed and crumbling (de Waal 2013: 154).  

The breakup of the Soviet Union consequently unleashed the complaints of Karabakh Armenians against these Azerbaijani government policies. The local government of Nagorno-Karabakh declared independence two days after Azerbaijan declared its independence from the Soviet Union on August 30, 1991. The vast majority of people in Nagorno-Karabakh voted yes in a referendum on independence on December 10. The Azeri government did not accept the secessionist move by Karabakh Armenians, resulting in the war of secession from 1992 to 1994.

The 1994 ceasefire line was drawn based on the territory that Karabakh's army occupied by the end of the war. It significantly differs from the Soviet-era boundary of Nagorno-Karabakh as Karabakh Armenians have occupied the large chunk of land in the south-west corner of Azerbaijan to be geographically contiguous with Armenia.

One consequence of the War of Secession is ethnic homogenization on both sides of the border. It is estimated that approximately 500,000 Azerbaijanis were displaced from the occupied part of Azerbaijan while 353,000 Armenians were displaced from the rest of Azerbaijan.%
\footnote{
	These numbers are based on Arif Yunusov's calculations, cited by de Waal (2013: 327).
}

Information is scant on economic conditions in Nagorno-Karabakh since the 1994 ceasefire. 
International Crisis Group (2005: 14-15) provides a rare account of Karabakh's economy as follows. 
\begin{itemize}
\item Entrepreneurs have been engaged in profitable scrap metal businesses by dismantling infrastructure, housing and other pre-war structures for the resale of metal, bricks and building materials in the occupied territory of Azerbaijan.
\item Privatization of land and state enterprises has largely been completed. All residents outside the capital city of Stepanakert received 0.6 hectares of arable land. Private companies account for 80\% of the output. 
\item Under-employment is widespread among rural residents although officials claim that the unemployment rate was 5.6\% in 2003. In 2005, the government started providing the unemployed with 60\% of the minimum wage.
\item The government encourages child birth through various benefits for children: large families receive free electricity and books; couples with three children or more receive cash grants; and those with six children or more are given new homes.
\item Income inequality is growing. In a survey of 1,000 by Stepanakert Press Club in October 2003, 40.5\% of the respondents answered that they lived better than five years ago while 57.6\% said their life got worse or didn't change.
\end{itemize}

On the other hand, Azerbaijan proper has experienced economic growth thanks to oil exports. 
In 2006, the oil pipeline was opened from Baku, the capital city, to Turkey via Georgia, contributing to the world's fastest economic growth from 2005 to 2008 (de Waal 2013: 291-292). 
World Bank (2006: Table 1) reports that the percentage of people living in poverty fell from 44.6\% in 2002 to 24.0\% in 2005.
This rapid decline is not concentrated in Baku and other urban areas but also seen in rural areas.
 
 \subsection{Abkhazia}
 International Alert (2004: 123) reports that GDP per capita in Abkhazia was roughly estimated to be \$350 in 2001, dropped from \$1333 in 1988, with the population dependent on subsistence farming and humanitarian aid for survival.
 
 \subsection{South Ossetia}
 Areshidze (2002), cited by International Crisis Group (2004: 11), estimates that GDP per capita in South Ossetia was \$250 in 2002.

\section{Data}
\subsection{Real GDP}
We use the Penn World Table 9.0 (Freenstra et al.\ 2015), as the data source for real GDP. 
Expenditure-side real GDP at chained PPPs in 2011 US dollars (the variable \textit{rgdpe}) is used because the variable is meant for comparing ``living standards across countries and across years'' (Freenstra et al.\ 2015, Table 1).

% Comparison to WDI
An alternative data source for real GDP is World Development Indicators (World Bank 2018). 
We prefer the Penn World Table, however, because since version 8 it makes real GDP cross-sectionally comparable in multiple years. 
The cross-sectional comparability in the World Development Indicators is only ensured for the latest international price survey year (2010 for the latest version); the farther away from this reference year, the less comparable its data on real GDP across countries (see Pinkovskiy and Sala-i-Martin 2016, sections 2.1-2.2, for detail).
As we exploit the variation in nighttime light not only within each country but also within each year, cross-sectional comparability is important.

\subsection{Nighttime light}
The data source for nighttime light is the DMSP-OLS Nighttime Lights Time Series, Version 4 (National Geophysical Data Center 2015). 
It provides annual panel data of nighttime light intensity at the 30 arc-second cells across the world. 
The value of light intensity ranges from 0 to 63, which is not comparable across years due to the aging and replacement of satellite sensors. 
See Henderson et al.\ (2012) among others for more detail about the dataset.
Following Hodler and Raschky (2014), who aggregate light data into the sub-national level, we take the average of all the cell-level values within each country/territory as a measure of light intensity at the country/territory level.

% Validation
Many studies validate light intensity against the measures of living standards.%
\footnote{See Mellander et al.\ (2015), Bickenbach et al.\ (2016), and Goldblatt et al.\ (2018) for recent validation exercises at the sub-national level.} 
For validation against real GDP at the country level, Henderson et al.\ (2012) and Storeygard (2016, table 1) show that the elasticity of real GDP with respect to light is around 0.3 across different subsamples of countries. 
In the estimation of elasticity, they control for country and year fixed effects, the latter of which take into account that the value of light intensity is not comparable across years.

\subsection{Country boundaries}
To aggregate nighttime light intensity at the 30 arc-second cell level into the country level, we need the spatial data on country boundaries. 
We use GADM (2018), following Alesina et al.\ (2016) and Dreher et al.\ (2015) among others. 
The data include 256 countries and overseas territories. 
One country, Vatican, is dropped from the sample because it is smaller than the 30 arc-second cell. 
In addition, 75 countries and territories are also dropped from the sample as real GDP data is unavailable for all years from 1992 to 2013.
Excluding the three parent states of unrecognized states (Georgia, Azerbaijan, and Moldova), our base sample thus consists of 177 countries/territories in total.

\subsection{Boundaries for unrecognized states}

\paragraph{Transnistria and Moldova proper} 
The boundaries of Transnistria and Moldova proper come from the first-level administrative boundary data from GADM (2018). 
The actual territory of Transnistria, however, differs from the official boundary between Transnistria and the rest of Moldova. 
Transnistria occupies some small parts of Moldova proper while Moldova controls some small parts of Transnistria (Tchepalyga\ 1997, p.\ 31). 
However, the spatial data on the exact boundary between the two is unavailable. 
Consequently, we use the official boundary as an approximation.

\paragraph{Abkhazia, South Ossetia and Georgia proper}  
The spatial data on the boundaries of Abkhazia, South Ossetia, and Georgia proper is obtained from ACASIAN (2014).%
\footnote{
	We do not use GADM (2018) because South Ossetia does not exist in the official subdivisions of Georgia.
	} 
This dataset was created in conjunction with the Institute of Geography at Russian Academy of Sciences, with the official map definitive as of the 1989 Soviet Census.

The Soviet-era boundaries accurately delineate the territories controlled by Abkhazia and South Ossetia after 2008 only, however. As a result of the wars of secession in 1991-1992, the north-east of Abkhazia (Upper Kodor Gorge) was under the control of Georgia proper (Tsutsiev 2014: Map 43). 
For South Ossetia, villages with Georgian residents were under the control of Georgia proper (Tsutsiev 2014: Map 42). 
In 2008, the defeat of Georgia against Russia resulted in the expulsion of the Georgian forces entirely from the Soviet-era territory of Abkhazia (Schwirtz 2008) and South Ossetia (O'Loughlin et al.\ 2014: 428).  

Since the data is unavailable for the pre-2008 boundaries of effectively controlled areas by Abkhazia and South Ossetia, this study uses the Soviet-era boundaries throughout the sample period of 1992-2013.

\paragraph{Nagorno-Karabakh and Azerbaijan proper}
The data on the internationally recognized boundary of Azerbaijan (which includes the western and southern boundaries of Nagorno-Karabakh) is also obtained from ACASIAN (2014). The data on the 1994 ceasefire line that separates Nagorno-Karabakh from Azerbaijan proper is created by the author, based on the map of Nagorno-Karabakh provided by the Armenian Ministry of Foreign Affairs (2018).

The 1994 ceasefire line acts as the boundary of effectively controlled areas of Nagorno-Karabakh until today. Consequently, the data used in this study correctly delineates Nagorno-Karabakh to the extent of accuracy in the original map and the digitization process by the author.

\section{Methodology}
% Model of real GDP
Following Henderson et al.\ (2012), we model real GDP in territory $i$ in year $t$, denoted by $y_{i,t}$, as follows:
\begin{equation}\label{gdp}
\ln y_{i,t} = \beta \ln L_{i,t} + \mu_i + \eta_t + \varepsilon_{i,t},
\end{equation}
where $L_{i,t}$ is nighttime light intensity in territory $i$ in year $t$, $\mu_i$ the territory fixed effect, $\eta_t$ the year fixed effect, and $\varepsilon_{i,t}$ the error term.

% Sample
To estimate $\beta$, $\mu_i$, and $\eta_t$ in equation (\ref{gdp}) with OLS, we use the balanced panel of countries around the world from 1992 to 2013 so that we can avoid compositional bias in the estimation of territory and year fixed effects, $\mu_i$ and $\eta_t$. 
The sample of countries excludes the parent countries of unrecognized states (i.e.\ Georgia, Azerbaijan, and Moldova) because we will validate our predicted values of real GDP against the actual real GDP in these three countries.
Standard errors are clustered both at the territory level and at the year level with the multi-way cluster robust inference method (Cameron et al.\ 2011).%
\footnote{Due to multicollinearity, we need to drop one dummy for either a country or a year when we estimate equation (\ref{gdp}). We drop the dummy for year 1992 because the country fixed effects estimates will be used to predict ``country'' fixed effects for unrecognized states and their parent countries, as described in the next paragraph.}

% Predict real GDP growth for unrecognized states
For unrecognized state $i$, we cannot predict $y_{i,t}$ from equation (\ref{gdp}) because there is no straightforward way to estimate $\mu_i$. However, we can predict the annual growth rate of $y_{i,t}$, denoted by $\hat{g}_{i,t}$, by taking the first-difference of equation (\ref{gdp}) to remove $\mu_i$:
\begin{eqnarray}\label{growth}
\hat{g}_{i,t} 
&=& \ln \hat{y}_{i,t} - \ln \hat{y}_{i, t-1} \nonumber \\
&=& \hat{\beta} (\ln L_{i,t} - \ln L_{i,t-1}) + (\hat{\eta}_t - \hat{\eta}_{t-1}),
\end{eqnarray}
where $\hat{\beta}$ and $\hat{\eta_t}$ are the estimated $\beta$ and $\eta_t$ in equation (\ref{gdp}).

\section{Results}
\subsection{Fixed Effects Estimation Results}
The results from estimating equation (\ref{gdp}) are reported in Table \ref{estimates}. 
The elasticity of real GDP with respect to nighttime light intensity is 0.35 (significant at the 1\% level).
This estimate is largely comparable to what the literature has found.%
\footnote{
	Henderson et al.\ (2012: Table 2) report the elasticity of 0.28 for the sample period of 1992-2008. Hodler and Raschky (2014: Appendix B) find the elasticity of 0.39 for subnational regions across the world for 1992-2009. Storeygard (2016: Table 1) obtains the elasticity of 0.25 for Chinese cities and prefectures for 1990-2005. 
}   
Year fixed effect estimates are also reported in Table \ref{estimates} as these values will be used to predict annual real GDP growth for unrecognized states. 
Note that the adjusted R-squared is 0.99. 
The model represented in equation (\ref{gdp}) explains almost all the variations in real GDP across countries and over time. 

Plugging the coefficient estimates in Table \ref{estimates} together with annual nighttime light observations into equation (\ref{growth}) yields the predicted growth of real GDP for unrecognized states and their parent countries. 
Figures \ref{aze} to \ref{tra} plot these predictions.

\subsection{Validation}
We first validate our methodology by comparing the predicted growth with the actual growth of real GDP from the Penn World Table for the three parent countries. 
% Azerbaijan
Figure \ref{aze} shows the results for Azerbaijan.
For years 1993-2004 and 2012-2013, the predicted growth rates do not align well with the actual growth rates, even showing a wrong sign in some years. 
From 2005 to 2011, however, the predicted growth rates capture the trend in the actual growth rates. 
The predicted growth generally underestimates the actual growth, probably because nighttime light does not capture the GDP growth due to oil exports.
% Georgia
Figure \ref{geo} compares the actual growth rates with the predicted ones for Georgia. 
Similarly to Azerbaijan, years 1993-2003 see discrepancies between the predicted and actual growth rates, with opposite signs in some years. 
For years 2004-2013, however, the predicted growth roughly picks up the trend in the actual growth, with the magnitude also comparable.
% Moldova
Figure \ref{mda} shows the results for Moldova.
For years 1993-2003 and 2011-2013, the predicted growth deviates from the actual growth, with a wrong sign in some years.
Years 2004-2010 see the predicted growth approximate the actual growth relatively well.
% Summary
Overall, the predicted growth around years 2004-2010 is trustworthy. Outside this period, either our prediction methodology fails or the official GDP statistics from these countries is prone to errors.

\subsection{Predicted Growth in Unrecognized States}
We now compare the predicted annual growth rates between unrecognized states and their parent countries. 
We use as a benchmark the predicted growth, not the actual growth, in the parent countries because the underlying data generating process is comparable.

\paragraph{Nagorno-Karabakh}
Figure \ref{nkr} shows the predicted annual real GDP growth rates for Nagorno-Karabakh in comparison to the predicted growth for its parent country, Azerbaijan. 
For years 1993-1995, Karabakh's economy is estimated to have shrunk by over 10\% annually, reflecting the damage from the war of secession from Azerbaijan. 
From 1996 to 2004, the estimated economic growth was very volatile.
For the period 2005-2011, where Azerbaijan's predicted growth rates trace the trend in the actual growth, Nagorno-Karabakh's economic growth exceeds 10\% (except for 2009 and 2011). 
These growth rates exceed the predicted growth rates in Azerbaijan proper.
It may just reflect the higher marginal product of capital per worker in Nagorno-Karabakh as the territory is poorer than the rest of Azerbaijan. 
However, it may also suggest that the impact of being unrecognized as a sovereign state is minimal in terms of economic recovery from civil wars in the case of Nagorno-Karabakh.

\paragraph{Abkhazia}
Figure \ref{abk} shows the predicted annual real GDP growth for Abkhazia in comparison to the predicted growth in Georgia proper. 
For years 1993-2003, the estimated economic growth rates are very volatile in Abkhazia though the ones in Georgia are equally volatile as well.
For the period 2004-2013, when Georgia's predicted growth corresponds well with its actual growth, Abkhazia's growth rates are positive and largely comparable to Georgia's growth.   
Overall, the impact of being unrecognized as a sovereign state appears to be minimal in the case of Abkhazia as long as Georgia proper is a valid counterfactual.

\paragraph{South Ossetia}
Figure \ref{sos} compares the predicted growth rates for South Ossetia with the ones for Georgia proper.
For the period 2004-2013, when Georgia's predicted and actual growth rates co-move, South Ossetia's predicted growth is comparable to Georgia's.
From 2010, South Ossetia outperforms Georgia in predicted growth, probably reflecting the consequences of South Ossetian War in 2008: Russia actively helps South Ossetia financially while Georgia loses its trade with Russia.

\paragraph{Transnistria}
Finally, Figure \ref{tra} reports the predicted economic growth rates for Transnistria in comparison to those for Moldova proper. 
Transnistria's predicted economic growth is volatile during the entire sample period, including years 2004-2010, when Moldova's predicted and actual real GDP growth rates correspond well to each other. 
Its magnitude is overall smaller than the other three unrecognized states.
The same remarks apply to its parent country, Moldova, suggesting that the impact of being unrecognized as a sovereign state is minimal in the case of Transnistria.

\section{Conclusions}


\begin{thebibliography}{99}
\bibitem{} ACASIAN (Australian Consortium for the Asian Spatial Information and Analysis Network). 2014. \textit{Russian Federation and Former Soviet Republics Research GIS Databases}. http://acasian.com (accessed on 20 February, 2018).
\bibitem{} Alesina, Alberto, Stelios Michalopoulos, and Elias Papaioannou. 2016. ``Ethnic Inequality.'' \textit{Journal of Political Economy}, 124(2): 428-488.
\bibitem{} Armenian Ministry of Foreign Affairs. 2018. ``Nagorno-Karabakh Republic (Artsakh).'' https://www.mfa.am/en/nagorno-karabakh-issue/ (accessed on 20 April, 2018).
\bibitem{} Mamuka Areshidze. 2002. ``Current Economic Causes of Conflict in Georgia.'' Unpublished report for UK Department for International Development (DFID).
\bibitem{} Bickenbach, Frank, Eckhardt Bode, Peter Nunnenkamp, and Mareike Soder. 2016. ``Night lights and regional GDP.'' \textit{Review of World Economics}, 152 (2), 425-447.                                                                                         
\bibitem{} Cameron, A. Colin, Jonah B. Gelbach, and Douglas L. Miller. 2011. ``Robust Inference With Multiway Clustering.'' \textit{Journal of Business \& Economic Statistics}, 29(2): 238-249.
\bibitem{} de Waal, Thomas. 2013. \textit{Black Garden: Armenia and Azerbaijan through Peace and War} (Revised Edition). New York: New York University Press.
\bibitem{} Dreher, Axel, Andreas Fuchs, Roland Hodler, Bradley C. Parks, Paul A. Raschky, and Michael J. Tierney. 2015. ``Aid on Demand: African Leaders and the Geography of China?s Foreign Assistance.'' CEPR Discussion Paper, 10170.
\bibitem{} Feenstra, Robert C., Robert Inklaar, and Marcel P. Timmer. 2015. ``The Next Generation of the Penn World Table.'' \textit{American Economic Review}, 105(10): 3150-3182.
\bibitem{} GADM. 2018. ``GADM Maps and Data (version 3.6).'' https://gadm.org (accessed February 7, 2019).
\bibitem{} Goldblatt, Ran, Kilian Heilmann, and Yonatan Vaizman. 2018. ``Can Medium-Resolution Satellite Imagery Measure Economic Activity at Small Geographies? Evidence from Landsat in Vietnam.'' \textit{World Bank Economic Review}, forthcoming.
\bibitem{} Henderson, J. Vernon, Adam Storeygard, and David N. Weil. 2012. ``Measuring Economic Growth from Outer Space.'' \textit{American Economic Review}, 102(2): 994-1028.
\bibitem{} Hodler, Roland, and Paul A. Raschky. 2014. ``Regional Favoritism.'' \textit{Quarterly Journal of Economics}, 129(2): 995-1033.
\bibitem{} International Alert. 2004. \textit{From War Economies to Peace Economies in the South Caucasus}. International Alert.
\bibitem{} International Crisis Group. 2004. ``Georgia: Avoiding War in South Ossetia.'' ICG Europe Report, no.\ 159.
\bibitem{} International Crisis Group. 2005. ``Nagorno-Karabakh: Viewing the Conflict from the Ground.'' Crisis Group Europe Report, no.\ 166.
\bibitem{} Mellander, Charlotta, Jose Lobo, Kevin Stolarick, and Zara Matheson. 2015. ``Night-Time Light Data: A Good Proxy Measure for Economic Activity?'' \textit{PloS one}, 10(10), e0139779.
\bibitem{} National Geophysical Data Center. 2015. ``Version 4 DMSP-OLS Nighttime Lights Time Series.'' National Oceanic and Atmospheric Administration. https://ngdc.noaa.gov/eog/dmsp/downloadV4composites.html (accessed February 6, 2019).
\bibitem{} National Scientific and Applied Center for Preventive Medicine. 2006. \textit{Moldova Demographic and Health Survey 2005}. Calverton, Maryland: ORC Macro.
\bibitem{} O'Loughlin, John, Vladimir Kolossov, and Gerard Toal. 2014. ``Inside the Post-Soviet de Facto States: A Comparison of Attitudes in Abkhazia, Nagorny Karabakh, South Ossetia, and Transnistria.'' \textit{Eurasian Geography and Economics}, 55(5): 423-456.
\bibitem{} Pinkovskiy, Maxim, and Xavier Sala-i-Martin. 2016. ``Newer Need Not be Better: Evaluating the Penn World Tables and the World Development Indicators Using Nighttime Lights.''. NBER Working Paper, 22216.
\bibitem{} Schwirtz, Michael. 2008. ``Abkhazia Wrests Gorge from Preoccupied Georgia.'' \textit{New York Times}, August 17, 2008.
\bibitem{} Storeygard, Adam. 2016. ``Farther on down the Road: Transport Costs, Trade and Urban Growth in Sub-Saharan Africa.'' \textit{The Review of Economic Studies}, 83(3): 1263-1295.
\bibitem{} Tchepalyga, A. L.\ 1997. \textit{Atlas of Dniester Moldavian Republic}. Tiraspol: Dniester State Corporative T.G.\ Shevchenko University.
\bibitem{} Tsutsiev, Arthur. 2014. \textit{Atlas of the Ethno-political History of the Caucasus}. New Haven: Yale University Press.
\bibitem{} Witmer, Frank D.W., and John O?Loughlin. 2011. ``Detecting the Effects of Wars in the Caucasus Regions of Russia and Georgia Using Radiometrically Normalized DMSP-OLS Nighttime Lights Imagery.'' \textit{GIScience \& Remote Sensing}, 48(4): 478-500.
\bibitem{} World Bank. 2006. ``Country Partnership Strategy FY07-10 for Republic of Azerbaijan.'' South Caucasus Country Management Unit, the World Bank.
\bibitem{} World Bank. 2018. \textit{World Development Indicators}. http://wdi.worldbank.org (accessed February 27, 2019).
\end{thebibliography}

\begin{figure}[ptb]
\includegraphics[width=\linewidth]{../a_output/plot_growth_hat_AZE.png}
\caption{Actual and Predicted Annual Real GDP Growth in Azerbaijan}
\label{aze}%
{\scriptsize \textbf{Notes}: 
	The data source for the actual annual real GDP growth is the Penn World Table (version 9.0). 
	The predicted annual real GDP growth is based on nighttime light intensity within Azerbaijan proper, excluding the territory occupied by Nagorno-Karabakh's authority. 
}
\end{figure}

\begin{figure}[ptb]
\includegraphics[width=\linewidth]{../a_output/plot_growth_hat_GEO.png}
\caption{Actual and Predicted Annual Real GDP Growth in Georgia}
\label{geo}%
{\scriptsize \textbf{Notes}: 
	The data source for the actual annual real GDP growth is the Penn World Table (version 9.0). 
	The predicted annual real GDP growth is based on nighttime light intensity within Georgia proper, excluding the Soviet-era territories of Abkhazia and South Ossetia. 
}
\end{figure}

\begin{figure}[ptb]
\includegraphics[width=\linewidth]{../a_output/plot_growth_hat_MDA.png}
\caption{Actual and Predicted Annual Real GDP Growth in Moldova}
\label{mda}%
{\scriptsize \textbf{Notes}: 
	The data source for the actual annual real GDP growth is the Penn World Table (version 9.0). 
	The predicted annual real GDP growth is based on nighttime light intensity within Moldova proper, excluding the Soviet-era territory of Transnistria. 
	The actual growth in 1993 is very close to zero, rather than being missing in the data source.
}
\end{figure}

\begin{figure}[ptb]
\includegraphics[width=\linewidth]{../a_output/plot_growth_hat_NKR.png}
\caption{Predicted Annual Real GDP Growth in Nagorno-Karabakh and Azerbaijan}
\label{nkr}%
{\scriptsize \textbf{Notes}: 
	The predicted annual real GDP growth for Nagorno-Karabakh is due to the author's calculation (see the text for detail).
	The corresponding data for Azerbaijan is the same as in Figure \ref{aze}.
}
\end{figure}

\begin{figure}[ptb]
\includegraphics[width=\linewidth]{../a_output/plot_growth_hat_ABK.png}
\caption{Predicted Annual Real GDP Growth in Abkhazia and Georgia}
\label{abk}%
{\scriptsize \textbf{Notes}: 
	The predicted annual real GDP growth for Abkhazia is due to the author's calculation (see the text for detail).
	The corresponding data for Georgia is the same as in Figure \ref{geo}.
}
\end{figure}

\begin{figure}[ptb]
\includegraphics[width=\linewidth]{../a_output/plot_growth_hat_SOS.png}
\caption{Predicted Annual Real GDP Growth in South Ossetia and Georgia}
\label{sos}%
{\scriptsize \textbf{Notes}: 
	The predicted annual real GDP growth for South Ossetia is due to the author's calculation (see the text for detail).
	The corresponding data for Georgia is the same as in Figure \ref{geo}.
}
\end{figure}

\begin{figure}[ptb]
\includegraphics[width=\linewidth]{../a_output/plot_growth_hat_TRA.png}
\caption{Predicted Annual Real GDP Growth in Transnistria and Moldova}
\label{tra}%
{\scriptsize \textbf{Notes}: 
	The predicted annual real GDP growth for Transnistria is due to the author's calculation (see the text for detail).
	The corresponding data for Moldova is the same as in Figure \ref{mda}.
}
\end{figure}

\begin{table}[ptb]
\caption{Estimated coefficients on mean light intensity and year dummies}%
\label{estimates}%
% Manually revised from ../a_output/income_light_cfe_yfe_result.tex
\begin{tabular*}{\textwidth}{@{\extracolsep\fill}lcclc} 
\\
[-1.8ex]
\hline \\
[-1.8ex] 
\multicolumn{5}{c}{Dependent Variable: Log real GDP} \\ 
\hline \\
[-1.8ex] 
 Log mean light 	& 0.344$^{***}$ 	& \ \  \ \ \ & Year 2003 	& 0.301$^{***}$ 	\\ 
  			& (0.064)   		& & 			& (0.035) 		\\  
[1.0ex]
 Year 1993 		& $-$0.040 	& & Year 2004 	& 0.349$^{***}$ 	\\  
  			& (0.037) 		& & 			& (0.044) 		\\ 
[1.0ex]
  Year 1994 	& $-$0.009 	& & Year 2005 	& 0.429$^{***}$	\\ 
  			& (0.035) 		& &			& (0.047) 		\\ 
[1.0ex] 
 Year 1995 		& $-$0.038 	& & Year 2006 	& 0.484$^{***}$ 	\\ 
 			& (0.037) 		& & 			& (0.052) 		\\ 
[1.0ex]
 Year 1996 		& 0.014 		& & Year 2007 	& 0.535$^{***}$ 	\\ 
 			& (0.033) 		& &			& (0.057)  		\\ 
[1.0ex]
 Year 1997 		& 0.089$^{***}$ 	& & Year 2008 	& 0.552$^{***}$	\\ 
 			& (0.026) 		& &			& (0.065)		 \\ 
[1.0ex] 
 Year 1998 		& 0.077$^{**}$ 	& & Year 2009 	& 0.575$^{***}$ 	\\ 
 			& (0.031) 		& &			& (0.059) 		\\ 
[1.0ex]
 Year 1999 		& 0.121$^{***}$ 	& & Year 2010 	& 0.486$^{***}$	\\ 
  			& (0.030) 		& & 			& (0.089)		\\ 
[1.0ex] 
 Year 2000 		& 0.153$^{***}$ 	& & Year 2011 	& 0.618$^{***}$	\\ 
  			& (0.037) 		& & 			& (0.080)		\\ 
[1.0ex] 
 Year 2001 		& 0.178$^{***}$ 	& & Year 2012 	& 0.628$^{***}$	\\ 
  			& (0.039) 		& & 			& (0.085)		\\ 
[1.0ex]			
 Year 2002 		& 0.208$^{***}$ 	& & Year 2013 	& 0.668$^{***}$	\\ 
  			& (0.044) 		& &			& (0.083)		 \\ 
[1.0ex]			
\hline \\
[-1.8ex] 
Country fixed effects & \multicolumn{4}{c}{Yes}  \\ 
Number of countries & \multicolumn{4}{c}{177}  \\ 
Observations & \multicolumn{4}{c}{3,894} \\ 
Adjusted R$^{2}$ & \multicolumn{4}{c}{0.988} \\ 
\hline 
\hline \\
[-1.5ex] % Remove the space between the bottom end of the table and the footnote below
\end{tabular*} 
\\
{\scriptsize \textbf{Notes}: 
	Estimated coefficients from equation (\ref{gdp}) are reported with standard errors clustered at the country and year levels in parentheses. *** indicates statistically significant at the 1\% level; ** 5\%.
}
\end{table}

\end{document}