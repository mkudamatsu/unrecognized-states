\documentclass[12pt,a4paper]{article}%
\usepackage{amsmath}
\usepackage{amsfonts}
\usepackage{amssymb}
\usepackage{graphicx}%
\setcounter{MaxMatrixCols}{30}
%TCIDATA{OutputFilter=latex2.dll}
%TCIDATA{Version=5.50.0.2953}
%TCIDATA{CSTFile=40 LaTeX article.cst}
%TCIDATA{Created=Saturday, March 08, 2008 16:13:47}
%TCIDATA{LastRevised=Monday, March 10, 2008 16:21:31}
%TCIDATA{<META NAME="GraphicsSave" CONTENT="32">}
%TCIDATA{<META NAME="SaveForMode" CONTENT="1">}
%TCIDATA{BibliographyScheme=Manual}
%TCIDATA{<META NAME="DocumentShell" CONTENT="Standard LaTeX\Blank - Standard LaTeX Article">}
%BeginMSIPreambleData
\providecommand{\U}[1]{\protect\rule{.1in}{.1in}}
%EndMSIPreambleData
\newtheorem{theorem}{Theorem}
\newtheorem{acknowledgement}[theorem]{Acknowledgement}
\newtheorem{algorithm}[theorem]{Algorithm}
\newtheorem{axiom}[theorem]{Axiom}
\newtheorem{case}[theorem]{Case}
\newtheorem{claim}[theorem]{Claim}
\newtheorem{conclusion}[theorem]{Conclusion}
\newtheorem{condition}[theorem]{Condition}
\newtheorem{conjecture}[theorem]{Conjecture}
\newtheorem{corollary}[theorem]{Corollary}
\newtheorem{criterion}[theorem]{Criterion}
\newtheorem{definition}[theorem]{Definition}
\newtheorem{example}[theorem]{Example}
\newtheorem{exercise}[theorem]{Exercise}
\newtheorem{lemma}[theorem]{Lemma}
\newtheorem{notation}[theorem]{Notation}
\newtheorem{problem}[theorem]{Problem}
\newtheorem{proposition}[theorem]{Proposition}
\newtheorem{remark}[theorem]{Remark}
\newtheorem{solution}[theorem]{Solution}
\newtheorem{summary}[theorem]{Summary}
\newenvironment{proof}[1][Proof]{\noindent\textbf{#1.} }{\ \rule{0.5em}{0.5em}}

\usepackage{bigfoot} % Allow \verb in a footnote

%%%%%%%%%%%%%%%%%%%%%%%%%%%%%%%%%%%%%%%%%%%%%%%%%%%%%%%%%%%%%%%%%%%%%%%%%%%%%%%%%%%%%%%
\begin{document}

\title{Estimating Output Loss due to being Unrecognized as States}
\date{}
\maketitle

\section{Introduction}
\section{Data}
\subsection{Real GDP}
We use the Penn World Table 9.0 (Freenstra et al.\ 2015), as the data source for real GDP. 
Expenditure-side real GDP at chained PPPs in 2011 US dollars (the variable \textit{rgdpe}) is used because the variable is meant for comparing ``living standards across countries and across years'' (Freenstra et al.\ 2015, Table 1).

% Comparison to WDI
An alternative data source for real GDP is World Development Indicators (World Bank 2018). 
We prefer the Penn World Table, however, because since version 8 it makes real GDP cross-sectionally comparable in multiple years. 
The cross-sectional comparability in the World Development Indicators is only ensured for the latest international price survey year (2010 for the latest version); the farther away from this reference year, the less comparable its data on real GDP across countries (see Pinkovskiy and Sala-i-Martin 2016, sections 2.1-2.2, for detail).
As we exploit the variation in nighttime light not only within each country but also within each year, cross-sectional comparability is important.

\section{Methodology}
% Model of real GDP
We model real GDP in territory $i$ in year $t$, denoted by $y_{it}$, as follows:
\begin{equation}\label{gdp}
y_{it} = \beta L_{it} + \mu_i + \eta_t + \varepsilon_{it},
\end{equation}
where $L_{it}$ is nighttime light intensity in territory $i$ in year $t$, $\mu_i$ the territory fixed effect, $\eta_t$ the year fixed effect, and $\varepsilon_{it}$ the error term.

% Sample
To estimate $\beta$, $\mu_i$, and $\eta_t$ in equation (\ref{gdp}) with OLS, we use the balanced panel of countries around the world from 1992 to 2013 so that we can avoid compositional bias in the estimation of territory and year fixed effects, $\mu_i$ and $\eta_t$. 
The sample of countries excludes the parent countries of unrecognized states (i.e.\ Georgia, Azerbaijan, and Moldova) because we will validate our predicted values of real GDP against the actual real GDP in these three countries.
Standard errors are clustered both at the territory level and at the year level with the multi-way cluster robust inference method (Cameron et al.\ 2011).

% Model of country fixed effects
To predict $y_{it}$ for unrecognized states (and their parent countries dropped from the sample) from equation (\ref{gdp}), we need to recover territory fixed effects, $\mu_i$. 
We predict them with the average light intensity over the sample period of 1992-2013. 
Specifically, with the same set of countries as the one for estimating equation (\ref{gdp}), we estimate the following equation with OLS:
\begin{equation}\label{country_fe}
\hat{\mu}_i = \alpha + \gamma \bar{L}_{i} + \xi_{i},
\end{equation}
where $\hat{\mu}_i$ is the estimated territory fixed effect from equation (\ref{gdp}), $\bar{L}_{i}$ the average light intensity in territory $i$ over the sample period, and $\xi_i$ the error term.

% Prediction of real GDP
Consequently, GDP per capita in unrecognized state $k$ in year $t$, denoted by $\hat{y}_{kt}$, will then be predicted as follows:
\begin{equation}
\hat{y}_{kt} = \hat{\beta} L_{kt} + [\hat{\alpha} + \hat{\gamma} \bar{L}_{i}]  + \hat{\eta_t},
\end{equation}
where $\hat{\beta}$ and $\hat{\eta_t}$ are the estimated $\beta$ and $\eta_t$ in equation (\ref{gdp}), $\hat{\alpha}$ and $\hat{\gamma}$ the estimated $\alpha$ and $\gamma$ in equation (\ref{country_fe}).

\section{Results}
\section{Conclusions}


\begin{thebibliography}{99}                                                                                                %
\bibitem{} Cameron, A. Colin, Jonah B. Gelbach, and Douglas L. Miller. 2011. ``Robust Inference With Multiway Clustering.'' \textit{Journal of Business \& Economic Statistics}, 29(2): 238-249.	
\bibitem{} Feenstra, Robert C., Robert Inklaar, and Marcel P. Timmer. 2015. ``The Next Generation of the Penn World Table.'' \textit{American Economic Review}, 105(10): 3150-3182.
\bibitem{} Pinkovskiy, Maxim, and Xavier Sala-i-Martin. 2016. ``Newer Need Not be Better: Evaluating the Penn World Tables and the World Development Indicators Using Nighttime Lights.''. NBER Working Paper, 22216.
\bibitem{} World Bank. 2018. \textit{World Development Indicators}. http://wdi.worldbank.org (accessed February 27, 2019).
\end{thebibliography}

\end{document}