\documentclass[12pt,a4paper]{article}%
\usepackage{amsmath}
\usepackage{amsfonts}
\usepackage{amssymb}
\usepackage{graphicx}%
\setcounter{MaxMatrixCols}{30}
%TCIDATA{OutputFilter=latex2.dll}
%TCIDATA{Version=5.50.0.2953}
%TCIDATA{CSTFile=40 LaTeX article.cst}
%TCIDATA{Created=Saturday, March 08, 2008 16:13:47}
%TCIDATA{LastRevised=Monday, March 10, 2008 16:21:31}
%TCIDATA{<META NAME="GraphicsSave" CONTENT="32">}
%TCIDATA{<META NAME="SaveForMode" CONTENT="1">}
%TCIDATA{BibliographyScheme=Manual}
%TCIDATA{<META NAME="DocumentShell" CONTENT="Standard LaTeX\Blank - Standard LaTeX Article">}
%BeginMSIPreambleData
\providecommand{\U}[1]{\protect\rule{.1in}{.1in}}
%EndMSIPreambleData
\newtheorem{theorem}{Theorem}
\newtheorem{acknowledgement}[theorem]{Acknowledgement}
\newtheorem{algorithm}[theorem]{Algorithm}
\newtheorem{axiom}[theorem]{Axiom}
\newtheorem{case}[theorem]{Case}
\newtheorem{claim}[theorem]{Claim}
\newtheorem{conclusion}[theorem]{Conclusion}
\newtheorem{condition}[theorem]{Condition}
\newtheorem{conjecture}[theorem]{Conjecture}
\newtheorem{corollary}[theorem]{Corollary}
\newtheorem{criterion}[theorem]{Criterion}
\newtheorem{definition}[theorem]{Definition}
\newtheorem{example}[theorem]{Example}
\newtheorem{exercise}[theorem]{Exercise}
\newtheorem{lemma}[theorem]{Lemma}
\newtheorem{notation}[theorem]{Notation}
\newtheorem{problem}[theorem]{Problem}
\newtheorem{proposition}[theorem]{Proposition}
\newtheorem{remark}[theorem]{Remark}
\newtheorem{solution}[theorem]{Solution}
\newtheorem{summary}[theorem]{Summary}
\newenvironment{proof}[1][Proof]{\noindent\textbf{#1.} }{\ \rule{0.5em}{0.5em}}

\usepackage{bigfoot} % Allow \verb in a footnote

%%%%%%%%%%%%%%%%%%%%%%%%%%%%%%%%%%%%%%%%%%%%%%%%%%%%%%%%%%%%%%%%%%%%%%%%%%%%%%%%%%%%%%%
\begin{document}

\title{Estimating Output Loss due to being Unrecognized as States}
\date{}
\maketitle

\section{Introduction}
\section{Data}
\subsection{Real GDP}
We use the Penn World Table 9.0 (Freenstra et al.\ 2015), as the data source for real GDP. 
Expenditure-side real GDP at chained PPPs in 2011 US dollars (the variable \textit{rgdpe}) is used because the variable is meant for comparing ``living standards across countries and across years'' (Freenstra et al.\ 2015, Table 1).

% Comparison to WDI
An alternative data source for real GDP is World Development Indicators (World Bank 2018). 
We prefer the Penn World Table, however, because since version 8 it makes real GDP cross-sectionally comparable in multiple years. 
The cross-sectional comparability in the World Development Indicators is only ensured for the latest international price survey year (2010 for the latest version); the farther away from this reference year, the less comparable its data on real GDP across countries (see Pinkovskiy and Sala-i-Martin 2016, sections 2.1-2.2, for detail).
As we exploit the variation in nighttime light not only within each country but also within each year, cross-sectional comparability is important.

\subsection{Nighttime light}
The data source for nighttime light is the DMSP-OLS Nighttime Lights Time Series, Version 4 (National Geophysical Data Center 2015). 
It provides annual panel data of nighttime light intensity at the 30 arc-second cells across the world. 
The value of light intensity ranges from 0 to 63, which is not comparable across years due to the aging and replacement of satellite sensors. 
See Henderson et al.\ (2012) among others for more detail about the dataset.
Following Hodler and Raschky (2014), who aggregate light data into the sub-national level, we take the average of all the cell-level values within each country/territory as a measure of light intensity at the country/territory level.

% Validation
Many studies validate light intensity against the measures of living standards.%
\footnote{See Mellander et al.\ (2015), Bickenbach et al.\ (2016), and Goldblatt et al.\ (2018) for recent validation exercises at the sub-national level.} 
For validation against real GDP at the country level, Henderson et al.\ (2012) and Storeygard (2016, table 1) show that the elasticity of real GDP with respect to light is around 0.3 across different subsamples of countries. 
In the estimation of elasticity, they control for country and year fixed effects, the latter of which take into account that the value of light intensity is not comparable across years.

\subsection{Country boundaries}
To aggregate nighttime light intensity at the 30 arc-second cell level into the country level, we need the spatial data on country boundaries. 
We use GADM (2018), following Alesina et al.\ (2016) and Dreher et al.\ (2015) among others. 
The data include 256 countries and overseas territories. 
One country, Vatican, is dropped from the sample because it is smaller than the 30 arc-second cell. 
In addition, 75 countries and territories are also dropped from the sample as real GDP data is unavailable for all years from 1992 to 2013.
Excluding the three parent states of unrecognized states (Georgia, Azerbaijan, and Moldova), our base sample thus consists of 177 countries/territories in total.
\section{Methodology}
% Model of real GDP
Following Henderson et al.\ (2012), we model real GDP in territory $i$ in year $t$, denoted by $y_{i,t}$, as follows:
\begin{equation}\label{gdp}
\ln y_{i,t} = \beta \ln L_{i,t} + \mu_i + \eta_t + \varepsilon_{i,t},
\end{equation}
where $L_{i,t}$ is nighttime light intensity in territory $i$ in year $t$, $\mu_i$ the territory fixed effect, $\eta_t$ the year fixed effect, and $\varepsilon_{i,t}$ the error term.

% Sample
To estimate $\beta$, $\mu_i$, and $\eta_t$ in equation (\ref{gdp}) with OLS, we use the balanced panel of countries around the world from 1992 to 2013 so that we can avoid compositional bias in the estimation of territory and year fixed effects, $\mu_i$ and $\eta_t$. 
The sample of countries excludes the parent countries of unrecognized states (i.e.\ Georgia, Azerbaijan, and Moldova) because we will validate our predicted values of real GDP against the actual real GDP in these three countries.
Standard errors are clustered both at the territory level and at the year level with the multi-way cluster robust inference method (Cameron et al.\ 2011).%
\footnote{Due to multicollinearity, we need to drop one dummy for either a country or a year when we estimate equation (\ref{gdp}). We drop the dummy for year 1992 because the country fixed effects estimates will be used to predict ``country'' fixed effects for unrecognized states and their parent countries, as described in the next paragraph.}

% Predict real GDP growth for unrecognized states
For unrecognized state $i$, we cannot predict $y_{i,t}$ from equation (\ref{gdp}) because there is no straightforward way to estimate $\mu_i$. However, we can predict the annual growth rate of $y_{i,t}$, denoted by $\hat{g}_{i,t}$, by taking the first-difference of equation (\ref{gdp}) to remove $\mu_i$:
\begin{eqnarray}\label{growth}
\hat{g}_{i,t} 
&=& \ln \hat{y}_{i,t} - \ln \hat{y}_{i, t-1} \nonumber \\
&=& \hat{\beta} (\ln L_{i,t} - \ln L_{i,t-1}) + (\hat{\eta}_t - \hat{\eta}_{t-1}),
\end{eqnarray}
where $\hat{\beta}$ and $\hat{\eta_t}$ are the estimated $\beta$ and $\eta_t$ in equation (\ref{gdp}).

\section{Results}
The results from estimating equation (\ref{gdp}) are reported in Table \ref{panel_results}. 
The elasticity of real GDP with respect to nighttime light intensity is xxx (significant at the x\% level). 
Year fixed effect estimates are also reported as these values will be used to predict real GDP for unrecognized states. 

Plugging the coefficient estimates in Table \ref{panel_results} together with annual nighttime light observations into equation (\ref{growth}) yields the predicted growth of real GDP for unrecognized states and their parent countries. 
Figures \ref{aze} to \ref{tra} plot these predictions.

\subsection{Validation}
We first validate our methodology by comparing the predicted growth with the actual growth of real GDP from the Penn World Table for the three parent countries. 
% Azerbaijan
Figure \ref{aze} shows the results for Azerbaijan.
For years 1993-2004 and 2012-2013, the predicted growth rates do not align well with the actual growth rates, even showing a wrong sign in some years. 
From 2005 to 2011, however, the predicted growth rates capture the trend in the actual growth rates. 
The predicted growth generally underestimates the actual growth, probably because nighttime light does not capture the GDP growth due to oil exports.
% Georgia
Figure \ref{geo} compares the actual growth rates with the predicted ones for Georgia. 
Similarly to Azerbaijan, years 1993-2003 see discrepancies between the predicted and actual growth rates, with opposite signs in some years. 
For years 2004-2013, however, the predicted growth roughly picks up the trend in the actual growth, with the magnitude also comparable.
% Moldova
Figure \ref{mda} shows the results for Moldova.
For years 1993-2003 and 2011-2013, the predicted growth deviates from the actual growth, with a wrong sign in some years.
Years 2004-2010 see the predicted growth approximate the actual growth relatively well.
% Summary
Overall, the predicted growth around years 2004-2010 is trustworthy. Outside this period, either our prediction methodology fails or the official GDP statistics from these countries is prone to errors.

\subsection{Predicted Growth in Unrecognized States}
We now compare the predicted annual growth rates between unrecognized states and their parent countries. 
We use as a benchmark the predicted growth, not the actual growth, in the parent countries because the underlying data generating process is comparable.

\paragraph{Nagorno-Karabakh}
Figure \ref{nkr} shows the predicted annual real GDP growth rates for Nagorno-Karabakh in comparison to the predicted growth for its parent country, Azerbaijan. 
For years 1993-1995, Karabakh's economy is estimated to have shrunk by over 10\% annually, reflecting the damage from the war of secession from Azerbaijan. 
From 1996 to 2004, the estimated economic growth was very volatile.
For the period 2005-2011, where Azerbaijan's predicted growth rates trace the trend in the actual growth, Nagorno-Karabakh's economic growth exceeds 10\% (except for 2009 and 2011). 
These growth rates exceed the predicted growth rates in Azerbaijan proper.
It may just reflect the higher marginal product of capital per worker in Nagorno-Karabakh as the territory is poorer than the rest of Azerbaijan. 
However, it may also suggest that the impact of being unrecognized as a sovereign state is minimal in terms of economic recovery from civil wars in the case of Nagorno-Karabakh.

\paragraph{Abkhazia}
Figure \ref{abk} shows the predicted annual real GDP growth for Abkhazia in comparison to the predicted growth in Georgia proper. 
For years 1993-2003, the estimated economic growth rates are very volatile in Abkhazia though the ones in Georgia are equally volatile as well.
For the period 2004-2013, when Georgia's predicted growth corresponds well with its actual growth, Abkhazia's growth rates are positive and largely comparable to Georgia's growth.   
Overall, the impact of being unrecognized as a sovereign state appears to be minimal in the case of Abkhazia as long as Georgia proper is a valid counterfactual.

\paragraph{South Ossetia}
Figure \ref{sos} compares the predicted growth rates for South Ossetia with the ones for Georgia proper.
For the period 2004-2013, when Georgia's predicted and actual growth rates co-move, South Ossetia's predicted growth is comparable to Georgia's.
From 2010, South Ossetia outperforms Georgia in predicted growth, probably reflecting the consequences of South Ossetian War in 2008: Russia actively helps South Ossetia financially while Georgia loses its trade with Russia.

\paragraph{Transnistria}
Finally, Figure \ref{tra} reports the predicted economic growth rates for Transnistria in comparison to those for Moldova proper. 
Transnistria's predicted economic growth is volatile during the entire sample period, including years 2004-2010, when Moldova's predicted and actual real GDP growth rates correspond well to each other. 
Its magnitude is overall smaller than the other three unrecognized states.
The same remarks apply to its parent country, Moldova, suggesting that the impact of being unrecognized as a sovereign state is minimal in the case of Transnistria.

\section{Conclusions}


\begin{thebibliography}{99}                                                                                                %
\bibitem{} Alesina, Alberto, Stelios Michalopoulos, and Elias Papaioannou. 2016. ``Ethnic Inequality.'' \textit{Journal of Political Economy}, 124(2): 428-488.
\bibitem{} Bickenbach, Frank, Eckhardt Bode, Peter Nunnenkamp, and Mareike Soder. 2016. ``Night lights and regional GDP.'' \textit{Review of World Economics}, 152 (2), 425-447.                                                                                         
\bibitem{} Cameron, A. Colin, Jonah B. Gelbach, and Douglas L. Miller. 2011. ``Robust Inference With Multiway Clustering.'' \textit{Journal of Business \& Economic Statistics}, 29(2): 238-249.
\bibitem{} Dreher, Axel, Andreas Fuchs, Roland Hodler, Bradley C. Parks, Paul A. Raschky, and Michael J. Tierney. 2015. ``Aid on Demand: African Leaders and the Geography of China?s Foreign Assistance.'' CEPR Discussion Paper, 10170.
\bibitem{} Feenstra, Robert C., Robert Inklaar, and Marcel P. Timmer. 2015. ``The Next Generation of the Penn World Table.'' \textit{American Economic Review}, 105(10): 3150-3182.
\bibitem{} GADM. 2018. ``GADM Maps and Data (version 3.6).'' https://gadm.org (accessed February 7, 2019).
\bibitem{} Goldblatt, Ran, Kilian Heilmann, and Yonatan Vaizman. 2018. ``Can Medium-Resolution Satellite Imagery Measure Economic Activity at Small Geographies? Evidence from Landsat in Vietnam.'' \textit{World Bank Economic Review}, forthcoming.
\bibitem{} Henderson, J. Vernon, Adam Storeygard, and David N. Weil. 2012. ``Measuring Economic Growth from Outer Space.'' \textit{American Economic Review}, 102(2): 994-1028.
\bibitem{} Hodler, Roland, and Paul A. Raschky. 2014. ``Regional Favoritism.'' \textit{Quarterly Journal of Economics}, 129(2): 995-1033.
\bibitem{} Mellander, Charlotta, Jose Lobo, Kevin Stolarick, and Zara Matheson. 2015. ``Night-Time Light Data: A Good Proxy Measure for Economic Activity?'' \textit{PloS one}, 10(10), e0139779.
\bibitem{} National Geophysical Data Center. 2015. ``Version 4 DMSP-OLS Nighttime Lights Time Series.'' National Oceanic and Atmospheric Administration. https://ngdc.noaa.gov/eog/dmsp/downloadV4composites.html (accessed February 6, 2019).
\bibitem{} Pinkovskiy, Maxim, and Xavier Sala-i-Martin. 2016. ``Newer Need Not be Better: Evaluating the Penn World Tables and the World Development Indicators Using Nighttime Lights.''. NBER Working Paper, 22216.
\bibitem{} Storeygard, Adam. 2016. ``Farther on down the Road: Transport Costs, Trade and Urban Growth in Sub-Saharan Africa.'' \textit{The Review of Economic Studies}, 83(3): 1263-1295.
\bibitem{} World Bank. 2018. \textit{World Development Indicators}. http://wdi.worldbank.org (accessed February 27, 2019).
\end{thebibliography}

\begin{figure}[ptb]
\caption{Actual and Predicted Annual Real GDP Growth in Azerbaijan}%
\label{aze}%
\includegraphics[width=\linewidth]{../a_output/plot_growth_hat_AZE.png}
{\scriptsize \textit{Notes}: 
	The data source for the actual annual real GDP growth is the Penn World Table (version 9.0). 
	The predicted annual real GDP growth is based on nighttime light intensity within Azerbaijan proper, excluding the territory occupied by Nagorno-Karabakh's authority. 
}
\end{figure}

\begin{figure}[ptb]
\caption{Actual and Predicted Annual Real GDP Growth in Georgia}%
\label{geo}%
\includegraphics[width=\linewidth]{../a_output/plot_growth_hat_GEO.png}
{\scriptsize \textit{Notes}: 
	The data source for the actual annual real GDP growth is the Penn World Table (version 9.0). 
	The predicted annual real GDP growth is based on nighttime light intensity within Georgia proper, excluding the Soviet-era territories of Abkhazia and South Ossetia. 
}
\end{figure}

\begin{figure}[ptb]
\caption{Actual and Predicted Annual Real GDP Growth in Moldova}%
\label{mda}%
\includegraphics[width=\linewidth]{../a_output/plot_growth_hat_MDA.png}
{\scriptsize \textit{Notes}: 
	The data source for the actual annual real GDP growth is the Penn World Table (version 9.0). 
	The predicted annual real GDP growth is based on nighttime light intensity within Moldova proper, excluding the Soviet-era territory of Transnistria. 
	The actual growth in 1993 is very close to zero, rather than being missing in the data source.
}
\end{figure}

\begin{figure}[ptb]
\caption{Predicted Annual Real GDP Growth in Nagorno-Karabakh and Azerbaijan}%
\label{nkr}%
\includegraphics[width=\linewidth]{../a_output/plot_growth_hat_NKR.png}
{\scriptsize \textit{Notes}: 
	The predicted annual real GDP growth for Nagorno-Karabakh is due to the author's calculation (see the text for detail).
	The corresponding data for Azerbaijan is the same as in Figure \ref{aze}.
}
\end{figure}

\begin{figure}[ptb]
\caption{Predicted Annual Real GDP Growth in Abkhazia and Georgia}%
\label{abk}%
\includegraphics[width=\linewidth]{../a_output/plot_growth_hat_ABK.png}
{\scriptsize \textit{Notes}: 
	The predicted annual real GDP growth for Abkhazia is due to the author's calculation (see the text for detail).
	The corresponding data for Georgia is the same as in Figure \ref{geo}.
}
\end{figure}

\begin{figure}[ptb]
\caption{Predicted Annual Real GDP Growth in South Ossetia and Georgia}%
\label{sos}%
\includegraphics[width=\linewidth]{../a_output/plot_growth_hat_SOS.png}
{\scriptsize \textit{Notes}: 
	The predicted annual real GDP growth for South Ossetia is due to the author's calculation (see the text for detail).
	The corresponding data for Georgia is the same as in Figure \ref{geo}.
}
\end{figure}

\begin{figure}[ptb]
\caption{Predicted Annual Real GDP Growth in Transnistria and Moldova}%
\label{tra}%
\includegraphics[width=\linewidth]{../a_output/plot_growth_hat_TRA.png}
{\scriptsize \textit{Notes}: 
	The predicted annual real GDP growth for Transnistria is due to the author's calculation (see the text for detail).
	The corresponding data for Moldova is the same as in Figure \ref{mda}.
}
\end{figure}

\begin{table}[ptb]
\caption{Estimated coefficients on mean light intensity and year dummies}%
\label{estimates}%
% Manually revised from ../a_output/income_light_cfe_yfe_result.tex
\begin{tabular*}{\textwidth}{@{\extracolsep\fill}lcclc} 
\\
[-1.8ex]
\hline \\
[-1.8ex] 
\multicolumn{5}{c}{Dependent Variable: Log real GDP} \\ 
\hline \\
[-1.8ex] 
 Log mean light 	& 0.344$^{***}$ 	& \ \  \ \ \ & Year 2003 	& 0.301$^{***}$ 	\\ 
  			& (0.064)   		& & 			& (0.035) 		\\  
[1.0ex]
 Year 1993 		& $-$0.040 	& & Year 2004 	& 0.349$^{***}$ 	\\  
  			& (0.037) 		& & 			& (0.044) 		\\ 
[1.0ex]
  Year 1994 	& $-$0.009 	& & Year 2005 	& 0.429$^{***}$	\\ 
  			& (0.035) 		& &			& (0.047) 		\\ 
[1.0ex] 
 Year 1995 		& $-$0.038 	& & Year 2006 	& 0.484$^{***}$ 	\\ 
 			& (0.037) 		& & 			& (0.052) 		\\ 
[1.0ex]
 Year 1996 		& 0.014 		& & Year 2007 	& 0.535$^{***}$ 	\\ 
 			& (0.033) 		& &			& (0.057)  		\\ 
[1.0ex]
 Year 1997 		& 0.089$^{***}$ 	& & Year 2008 	& 0.552$^{***}$	\\ 
 			& (0.026) 		& &			& (0.065)		 \\ 
[1.0ex] 
 Year 1998 		& 0.077$^{**}$ 	& & Year 2009 	& 0.575$^{***}$ 	\\ 
 			& (0.031) 		& &			& (0.059) 		\\ 
[1.0ex]
 Year 1999 		& 0.121$^{***}$ 	& & Year 2010 	& 0.486$^{***}$	\\ 
  			& (0.030) 		& & 			& (0.089)		\\ 
[1.0ex] 
 Year 2000 		& 0.153$^{***}$ 	& & Year 2011 	& 0.618$^{***}$	\\ 
  			& (0.037) 		& & 			& (0.080)		\\ 
[1.0ex] 
 Year 2001 		& 0.178$^{***}$ 	& & Year 2012 	& 0.628$^{***}$	\\ 
  			& (0.039) 		& & 			& (0.085)		\\ 
[1.0ex]			
 Year 2002 		& 0.208$^{***}$ 	& & Year 2013 	& 0.668$^{***}$	\\ 
  			& (0.044) 		& &			& (0.083)		 \\ 
[1.0ex]			
\hline \\
[-1.8ex] 
Country fixed effects & \multicolumn{4}{c}{Yes}  \\ 
Number of countries & \multicolumn{4}{c}{177}  \\ 
Observations & \multicolumn{4}{c}{3,894} \\ 
Adjusted R$^{2}$ & \multicolumn{4}{c}{0.988} \\ 
\hline 
\hline \\
[-1.5ex] % Remove the space between the bottom end of the table and the footnote below
\end{tabular*} 
\\
{\scriptsize \textit{Notes}: 
	Estimated coefficients from equation (\ref{gdp}) are reported with standard errors clustered at the country and year levels in parentheses. *** indicates statistically significant at the 1\% level; ** 5\%.
}
\end{table}

\end{document}